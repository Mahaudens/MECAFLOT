
\documentclass[a4paper,12pt]{article}


% --- LISTE DES PACKAGES DE BASE ----
% - GESTION DE DOCUMENT -
\usepackage[top=1in, bottom=0.75in, left=1.25in, right=1.25in]{geometry}	
% Permet de changer les marges
\usepackage{comment}
% Permet d'ajouter des commentaires
\usepackage[toc,page]{appendix}
% Gestion de numérotation de section
\usepackage{hyperref}
% Permet de faire des liens hypertexte
\usepackage{cite}
% Permet de citer
\usepackage{url}
% Permet de mieux gérer les liens url
\usepackage{color, colortbl}
% Permet l'utilisation de couleurs
%\usepackage{pslatex}
% Permet d'utiliser un style à la Delhez
\usepackage{ifthen}
% Gestion de conditions et boucles
\usepackage[nottoc,notlot,notlof]{tocbibind}
% Permet de metter la bibliographie dans la toc
\usepackage{fancyhdr}
% Permet de mettre des en-têtes et bas de pages

% - GESTION DE TEXTE -
\usepackage[T1]{fontenc}		
%\renewcommand*\familydefault{\sfdefault}				


% Permet les accents
\usepackage[utf8]{inputenc}					
% Gestion d'encodage
\usepackage[francais]{babel}					
% Gestion de typographie française
\usepackage{babelbib}
% Gestion de typographie internationale
\frenchbsetup{StandardLists=true} 		
% Permet d'éviter les conflits avec le package enumitem
\usepackage{enumitem}							
% Gestion de liste
\usepackage{multirow, multicol}
% Permet de créer des rangées au sein de rangées
\usepackage{lscape}
% Gestion de tableaux sur plusieurs pages
\usepackage{verbatim}
% Permet d'afficher du LaTex sans l'exécuter
\usepackage{booktabs}
% Gestion de tableaux améliorée
\usepackage{textcomp}
% Gestion de symboles pour du texte
\usepackage{lipsum}  
% Permet de générer du texte latin

% - GESTION DE RESSOURCES MATHEMATIQUES -
\usepackage{amsmath, amssymb,amsthm}		
% Gestion de symboles mathématiques
\usepackage{thmbox}										
% Gestion de structures de théorèmes
\usepackage{fancybox}
% Permet l'utilisation de cadres
\usepackage{bm}
% Permet d'utiliser du gras pour les symboles mathématiques

% - GESTION GRAPHIQUE - 
\usepackage{graphicx}									
% Permet l'import de figures
\usepackage{float}
% Gestion d'environnements d'objets flottants
\usepackage[section]{placeins}
% Permet de stopper les objets flottants entre les frontières de sections
\usepackage{tikz}
% Gestion de dessins
\usepackage{tikz-3dplot}
% Gestion de dessins 3D
\usepackage{pgfplots}
% Permet de faire des graphiques
\usepackage{pgfplotstable}
% Permet de convertir des données d'un fichier externe en tableau
\usetikzlibrary{intersections,calc,angles,quotes,through}
% Import de librarires tikz pour dessiner (1/2)
\usetikzlibrary{decorations.markings,patterns,shapes,arrows,positioning}
% Import de librarires tikz pour dessiner (2/2)
\usepackage[hang]{caption}
% Permet l'utilisation de descriptions en dessous d'objets
\usepackage{subcaption}
% Permet l'utilisation de descriptions en dessous de subobjets

% --- PARAMETRES ---
% On renome les "tables" en "tableaux"
\addto\captionsfrench{\def\tablename{\textsc{Tableau}}}
% On définit l'espace entre paragraphes
\setlength{\parskip}{1em}
% On paramètre les en-têtes et bas de pages 
\pagestyle{fancy}
\fancyhf{}
\fancyhead[R]{\rightmark}
\fancyfoot[C]{\thepage}


\usepackage{minitoc}


\begin{document}

\renewcommand{\contentsname}{Table des matières}
\renewcommand{\listfigurename}{Liste des figures}
\renewcommand{\listtablename}{Liste des tableaux}

\renewcommand{\thesection}{\arabic{section}.}
\renewcommand{\thesubsection}{\arabic{section}.\arabic{subsection}.}
\renewcommand{\thesubsubsection}{\arabic{section}.\arabic{subsection}.\arabic{subsubsection}.}


% --- PAGE DE GARDE ET TABLE DES MATIERES		
\thispagestyle{empty}
\begin{center}
	\Large Université de Liège\\
	Faculté des Sciences Appliquées \\
	\vspace{0.5cm}
	\includegraphics[width=8cm]{ulg.jpg}
\end{center}
\vspace*{3cm}
 \begin{center}
	\noindent\rule[0.5ex]{\textwidth}{1pt}
	\textsc{\large MECA0011-2: Eléments de mécanique des fluides}\\
	\vspace*{0.1cm}
	\bfseries \Huge Simulation numérique d'un écoulement
 	\vspace*{-0.6cm}
	\noindent\rule[0.5ex]{\textwidth}{1pt}
\end{center}
\centerline{\large Yann {\sc Drèze}   \quad Thomas {\sc Gregov} \quad Julien {\sc Hindryckx}}
\vfill
\begin{center}
	\large\textsl{Bachelier en Sciences de l'Ingénieur, orientation Ingénieur Civil }\\
	\vspace*{0.5cm}
	Année académique 2017-2018
\end{center}
\pagebreak
\newpage
\newgeometry{headheight=15.2pt, top=1in, bottom=1.25in, left=1.25in, right=1.25in}
\fancyhead[R]{}

\doparttoc
\tableofcontents

\newpage
\fancyhead[R]{}
\listoffigures
\listoftables
\newpage
\renewcommand{\sectionmark}[1]{\markboth{#1}{}}
\fancyhead[LO, LE]{{\fontfamily{cmss}\selectfont \MakeUppercase{ \leftmark}}}
\fancyhead[RO, RE]{{\fontfamily{cmss}\selectfont\rightmark}}

\mtcsettitle{parttoc}{Table des matières salut}


\doparttoc
% --- DEBUT DU CORPS PRINCIPAL DU TEXTE --
\part{Modélisation}
\parttoc

\section{Principe de l'expérience}
\subsection{Test}
	\lipsum[1]

	\begin{figure}[H]
		\centering
		\begin{subfigure}{.49\textwidth}
			\centering
			\begin{tikzpicture}
				\begin{axis}[
    				xlabel={$\Delta V_{\text{multimètre $I_A$}}$ [V]},
    				ylabel={$I_A$ [nA]},
    				width=\textwidth,
    				height=\textwidth,
    				axis lines=center,
    				enlarge x limits= 0.15,
    				enlarge y limits= 0.15,
    				grid,
    				every axis/.append style={very thick},
    				legend style={at={(1,0.15)}, anchor =south east,  cells={align=right}},
    				no markers]
   
    				\pgfplotstableread{IVprop.txt}{\loadedtable}
					\pgfplotstablecreatecol[
    					create col/expr={\thisrow{V}}]{V}{\loadedtable}
    				\pgfplotstablecreatecol[linear regression]{regression}{\loadedtable}
					\pgfplotstablecreatecol[create col/expr={\thisrow{V}-\thisrow{regression}}]{error}{\loadedtable}
					\pgfmathsetmacro\totalsquarederror{0}
					\pgfplotstableforeachcolumnelement{error}\of\loadedtable\as\error{
						\pgfmathsetmacro\totalsquarederror{\totalsquarederror+(\error)^2}}
					\pgfplotstablegetrowsof\loadedtable
					\pgfmathsetmacro\meansquarederror{\totalsquarederror/\pgfplotsretval}

					\def \newCommandName {$\pgfmathprintnumber{\h}$}
   						\addplot [dashed, blue] table [
     						x=V,
    						y={create col/linear regression={y=I}}]{\loadedtable};
      					\addplot [only marks] table [
   					 		x=V,
    						y=I] {\loadedtable};
				\end{axis}
			\end{tikzpicture}
			\caption{Cas de la mesure de $I_A$.}
			\label{fig:regcoeffalpha}
		\end{subfigure}
		\begin{subfigure}{.49\textwidth}
 			\centering
			\begin{tikzpicture}
				\begin{axis}[
   	 				xlabel={$\Delta V_{\text{multimètre $U_1$}}$ [V]},
    				ylabel={$U_1$ [V]},
    				width=\textwidth,
    				height=\textwidth,
    				axis lines=center,
    				enlarge x limits= 0.15,
    				enlarge y limits= 0.15,
    				grid,
    				every axis/.append style={very thick},
    				legend style={at={(1,0.15)}, anchor =south east,  cells={align=right}},
					no markers]
   
    				\pgfplotstableread{VVprop.txt}{\loadedtable}
					\pgfplotstablecreatecol[create col/expr={\thisrow{V}}]{V}{\loadedtable}
					\pgfplotstablecreatecol[linear regression]{regression}{\loadedtable}
					\pgfplotstablecreatecol[create col/expr={\thisrow{V}-\thisrow{regression}} ]{error}{\loadedtable}
					\pgfmathsetmacro\totalsquarederror{0}
					\pgfplotstableforeachcolumnelement{error}\of\loadedtable\as\error{
    					\pgfmathsetmacro\totalsquarederror{\totalsquarederror+(\error)^2}}
					\pgfplotstablegetrowsof\loadedtable
					\pgfmathsetmacro\meansquarederror{\totalsquarederror/\pgfplotsretval}

					\def \newCommandName {$\pgfmathprintnumber{\h}$}
  	 				\addplot [dashed, blue] table [
     					x=V,
    					y={create col/linear regression={y=VUC}}]{\loadedtable};
     
      				\addplot [only marks] table [
    					x=V,
    					y=VUC] {\loadedtable};
				\end{axis}
			\end{tikzpicture}
			\caption{Cas de la mesure de $U_1$.}
			\label{fig:regcoeffbeta}
		\end{subfigure}
		\caption{Régressions linéaires entre les valeurs mesurées $\Delta V_{\text{multimètre $I_A$}}$ (resp. $\Delta V_{\text{multimètre $I_A$}}$) et les grandeurs appliquées $I_A$ (resp. $U_1$).}
		\label{fig:regcoeff}
	\end{figure}
	
	\lipsum[3]

\subsection{Dispositif}
	\lipsum[1]
	
	\begin{figure}[H]
  		\centering
		\begin{tikzpicture}
			\begin{axis}[
   				xlabel={$U_1$ [V]},
    			ylabel={$I_A$ [nA]},
    			width=\textwidth,
   			 	height=0.6\textwidth,
    			axis lines=center,
    			enlarge x limits= 0.1,
    			enlarge y limits= 0.15,
    			legend style={at={(1,0.55)}, anchor =south east,  cells={align=right}},
    			grid style = dashed,
    			every axis/.append style={very thick},
    			no markers,
    			grid]
    
				\addplot table [
    				x=U1,
    				y=IA,
    				white space chars={{,},\ },
            		x expr=\thisrowno{0}.\thisrowno{1}, y expr=\thisrowno{2}.\thisrowno{3}] {180.txt};
			\end{axis}
		\end{tikzpicture}
		\caption{Évolution du courant mesuré $I_A$ en fonction du potentiel appliqué $U_1$ à $T_{\text{four}}=180^\circ\,$C. Une interpolation linéaire a été appliquée sur les mesures.}
		\label{fig:ordi180}
	\end{figure} 

	\lipsum[8]

	\begin{table}[h]
		\centering
		\makebox[\textwidth]{
			\begin{tabular}{| c | c | c | c | c | c | c |}
				\hline
				$q_{\text{exp}}\times 10^{-19}$ [C] & $q_{\text{exp}}/q_{\text{ref}}$ & $2\cdot q_{\text{exp}}/q_{\text{ref}}$ & $3\cdot q_{\text{exp}}/q_{\text{ref}}$ & Rapport entier & $q_{\text{elem}}\times 10^{19}$ [C]\\ 
				 \hline
    			2.036  &  0.536&  1.0723 &  1.608 & /  &   /\\
    			3.798  & 1.000    &2.000   &  3.000  &3   & 1.266\\
    			4.000 &    1.000 &     2.107 &  3.160 &  3    & 1.333\\
     			6.847 &     1.803 &    3.606& 5.409&   /  & /\\
 			   	7.578 &   1.999 &   3.990 &   5.985&  6 & 1.263\\
 			   	8.526 &    2.245&   4.490 &   6.735 & 7   & 1.218 \\
 			   	9.097&   2.395&   4.791 &   7.186  & 7    &  1.300\\
 				9.518 &     2.506&    5.013 &   7.519 & /   &  /\\
   			  	10.523 &     2.771 &    5.542 &  8.312  & 8   & 1.315\\
   				10.986 & 2.893 &    5.789 &   8.679 & 9  & 1.221\\
     			11.159&     2.938&    5.877 &   8.815 & 9        & 1.240\\
    			12.554&    3.308 &    6.612 &   9.917  &10     & 1.255\\
    			14.167&    3.730&    7.461&  11.191   &11       &  1.288\\
    			17.564&     4.625&    9.250&   13.875  & 14    & 1.255 \\
    			18.401&     4.846&    9.691&  14.537 & /        & /\\
    			22.936&     6.040&     12.079&   18.119 &18     &    1.274\\
    			30.728&     8.091&    16.183&   24.274 & 24       &  1.280\\
    			37.935&      9.989&    19.979 &   29.968 & 30      &   1.265\\
    			\hline
			\end{tabular}}
		\caption{Charges expérimentales, rapports de charges et charges élementaires calculées, où $q_{\text{ref}}=3.798\times 10^{-19}\,$C.\label{tab:charges}}
	\end{table}
	
	BOOOOOOOOOOO
	Test de citation selon \cite{AbedonHymanThomas2003} et aussi\footnote{Test \cite{Nobody06}.} \cite{Abedon1994}.
	Test d'équation selon Yann \ref{fig:ordi180}, l'équation \eqref{eq:integ} d'interprétation de Riemann:
	\begin{equation}
		\int_a^bf(x)dx = \lim_{n\rightarrow + \infty}\sum_{k=0}^{n-1}\left[ f\left(\xi_k\right)\cdot (x_{k+1}-x_k)\right], \quad\xi_k \in [x_k, x_{k+1}]  \label{eq:integ}
	\end{equation}
	avec $a = x_0 < ... < x_n = b$ et $\sup_{k \in \{0,...,  n-1\}}( x_{k+1}-x_{k}) \overset{n \rightarrow + \infty}{\longrightarrow} 0$.

% --- FIN DU CORPS PRINCIPAL DU TEXTE --
% --- BIBLIOGRAPHIE ---
\newpage
\fancyhead[L]{}
\bibliographystyle{plain}
\bibliography{ref}


\end{document}